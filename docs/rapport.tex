\documentclass[12pt]{article}

\setlength{\parskip}{1.1em}
\setlength{\parindent}{0em}
\usepackage{pdfpages}
\usepackage{graphicx}
\usepackage{pgf}
\usepackage{amsmath}
\usepackage{listings}
\usepackage[T1]{fontenc}
\usepackage[utf8]{inputenc}
\usepackage{float}
\usepackage{graphicx}
\usepackage{cite}
\usepackage{times}
\usepackage{hyperref}
\usepackage{titlesec}
\usepackage{verbatim}
\titlespacing\section{0pt}{12pt plus 4pt minus 2pt}{0pt plus 2pt minus 2pt}
\titlespacing\subsection{0pt}{12pt plus 4pt minus 2pt}{0pt plus 2pt minus 2pt}
\titlespacing\subsubsection{0pt}{12pt plus 4pt minus 2pt}{0pt plus 2pt minus 2pt}
\usepackage{color}
\usepackage{courier}

\lstset{basicstyle=\footnotesize\ttfamily,breaklines=true}
\lstset{framextopmargin=50pt,frame=bottomline}


\lstset{ %
  backgroundcolor=\color{white},   % choose the background color; you must add \usepackage{color} or \usepackage{xcolor}
  basicstyle=\footnotesize,        % the size of the fonts that are used for the code
  breakatwhitespace=false,         % sets if automatic breaks should only happen at whitespace
  breaklines=true,                 % sets automatic line breaking
  captionpos=b,                    % sets the caption-position to bottom
  commentstyle=\color{magenta},    % comment style
  deletekeywords={...},            % if you want to delete keywords from the given language
  escapeinside={\%*}{*)},          % if you want to add LaTeX within your code
  extendedchars=true,              % lets you use non-ASCII characters; for 8-bits encodings only, does not work with UTF-8
  frame=single,	                   % adds a frame around the code
  keepspaces=true,                 % keeps spaces in text, useful for keeping indentation of code (possibly needs columns=flexible)
  keywordstyle=\color{blue},       % keyword style
  language=Python,                 % the language of the code
  otherkeywords={*,...},           % if you want to add more keywords to the set
  numbers=left,                    % where to put the line-numbers; possible values are (none, left, right)
  numbersep=5pt,                   % how far the line-numbers are from the code
  numberstyle=\tiny\color{gray},   % the style that is used for the line-numbers
  rulecolor=\color{black},         % if not set, the frame-color may be changed on line-breaks within not-black text (e.g. comments (green here))
  showspaces=false,                % show spaces everywhere adding particular underscores; it overrides 'showstringspaces'
  showstringspaces=false,          % underline spaces within strings only
  showtabs=false,                  % show tabs within strings adding particular underscores
  stepnumber=2,                    % the step between two line-numbers. If it's 1, each line will be numbered
  stringstyle=\color{magenta},     % string literal style
  tabsize=2,	                   % sets default tabsize to 2 spaces
  title=\lstname                   % show the filename of files included with \lstinputlisting; also try caption instead of title
}

\title{Fundamentals of Artificial Intelligence (57205HT16): Assignment 2 -
Happy, Sad, Mischievous or Mad?}
\author{
    Marc Coquand \\ 
    id14mcd \\
    mcoquand@gmail.com \and
		Linus Lagerhjelm \\
		id14llm \\
		id14llm@cs.umu.se \and \\
		Supervisor: Thomas Johansson\\
		Alexander Sutherland \\
		Thomas Hellström
}
\date{\today}

\renewcommand{\baselinestretch}{1.0}
\begin{document}
\maketitle

\newpage
\tableofcontents

\newpage
\section{Introduction}

This report contains the documentation and explanation of a face recognition
program. The program can recognize if a set of drawn faces are happy, sad,
mischievous or mad.


\subsection{Running the program}

The implementation is written in python 2.7 which means that a python 2.7 shell
is required to run the implementation. Required program arguments are a search
path to a \texttt{training.txt}, search path to an answer file
\texttt{training-facit.txt} and a test file \texttt{test-file.txt}. The format
of the training file should be the id of the image followed by the image on a
new line. The answer file should be formatted as the image id follow by the
correct answer (1,2,3 or 4).

Should the user fail to provide the required arguments, the program will
immediately terminate with a massage explaining that not enough input
arguments were provided.

Example usage of the program might look like:
\begin{verbatim}
python faces.py training-file.txt training-facit.txt test-file.txt
\end{verbatim}



\bibliographystyle{abbrv}
\bibliography{rapport}

\section {Problem description}

\subsection {Perceptron Learning Algorithm}
The program written implements an algorithm for supervised learning called
\textit{Perceptron Learning}. \textit{Supervised Learning} is a technuque
for machine learning where the ANN is exposed to input and produces a guess
based on the knowledge aquired from previous guesses. Since the desired
output from the ANN, in supervised learning, is known for every input, the
ANN is slightly modified with each iteration to eventually produce good
outputs for every set of inputs.

The ANN in the produced program is composed by four different Perceptrons
where each one represents one of the different facial expressions that are
to be identified in the assignment.

Each perceptron is implemented in \textit{Python} as a class composed by a
list of \textit{401} weights that mapps to the \textit{400} pixel values
that makes up an image. The additional weight is provided as a bias for
the perceptron. A bias is used in order to shift the result of the activation
function (more on that later) towards either the left or the right. I.e.
to make the output more extreme so that it will produce a more distinct
answer.

Training of the network occurs in the \textit{Tutor class} where the provided
set of training images is repetedly shown for each perceptron. The output
from a perceptron, also known as \textit{activation}, is computed using
the formula:

% Make pretty math notation here plz Marc
\begin{equation*}
  a_i = act(\sum_j x_j*w_{j,i})
\end{equation*}

where x is the input, w is the weight and act is the activation function
which is the sigmoid function:

\begin{equation*}
sigmoid(x) = 1/(1+e^{-x})
\end{equation*}

\begin{align*}
  &e_i = y_i - a_i\\
  &\nabla w_{j,i} = \alpha e_ix_j\\
  &w_{j,i} \leftarrow w_{j,i} + \nabla w_{j,i}
\end{align*}
  

\newpage
% This may not be the prettiest solution, but it's the one you deserve Linus
%\lstinputlisting[language=Python]{../src/given.py}
\end{document}
