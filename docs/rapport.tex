\documentclass[12pt]{article}

\setlength{\parskip}{1.1em}
\setlength{\parindent}{0em}
\usepackage{pdfpages}
\usepackage{graphicx}
\usepackage{pgf}
\usepackage{amsmath}
\usepackage{listings}
\usepackage[T1]{fontenc}
\usepackage[utf8]{inputenc}
\usepackage{float}
\usepackage{graphicx}
\usepackage{cite}
\usepackage{times}
\usepackage{hyperref}
\usepackage{titlesec}
\usepackage{verbatim}
\titlespacing\section{0pt}{12pt plus 4pt minus 2pt}{0pt plus 2pt minus 2pt}
\titlespacing\subsection{0pt}{12pt plus 4pt minus 2pt}{0pt plus 2pt minus 2pt}
\titlespacing\subsubsection{0pt}{12pt plus 4pt minus 2pt}{0pt plus 2pt minus 2pt}
\usepackage{color}
\usepackage{courier}

\lstset{basicstyle=\footnotesize\ttfamily,breaklines=true}
\lstset{framextopmargin=50pt,frame=bottomline}


\lstset{ %
  backgroundcolor=\color{white},   % choose the background color; you must add \usepackage{color} or \usepackage{xcolor}
  basicstyle=\footnotesize,        % the size of the fonts that are used for the code
  breakatwhitespace=false,         % sets if automatic breaks should only happen at whitespace
  breaklines=true,                 % sets automatic line breaking
  captionpos=b,                    % sets the caption-position to bottom
  commentstyle=\color{magenta},    % comment style
  deletekeywords={...},            % if you want to delete keywords from the given language
  escapeinside={\%*}{*)},          % if you want to add LaTeX within your code
  extendedchars=true,              % lets you use non-ASCII characters; for 8-bits encodings only, does not work with UTF-8
  frame=single,	                   % adds a frame around the code
  keepspaces=true,                 % keeps spaces in text, useful for keeping indentation of code (possibly needs columns=flexible)
  keywordstyle=\color{blue},       % keyword style
  language=Python,                 % the language of the code
  otherkeywords={*,...},           % if you want to add more keywords to the set
  numbers=left,                    % where to put the line-numbers; possible values are (none, left, right)
  numbersep=5pt,                   % how far the line-numbers are from the code
  numberstyle=\tiny\color{gray},   % the style that is used for the line-numbers
  rulecolor=\color{black},         % if not set, the frame-color may be changed on line-breaks within not-black text (e.g. comments (green here))
  showspaces=false,                % show spaces everywhere adding particular underscores; it overrides 'showstringspaces'
  showstringspaces=false,          % underline spaces within strings only
  showtabs=false,                  % show tabs within strings adding particular underscores
  stepnumber=2,                    % the step between two line-numbers. If it's 1, each line will be numbered
  stringstyle=\color{magenta},     % string literal style
  tabsize=2,	                   % sets default tabsize to 2 spaces
  title=\lstname                   % show the filename of files included with \lstinputlisting; also try caption instead of title
}

\title{Fundamentals of Artificial Intelligence (57205HT16): Assignment 2 -
Happy, Sad, Mischievous or Mad?}
\author{
    Marc Coquand \\ 
    id14mcd \\
    mcoquand@gmail.com \and
		Linus Lagerhjelm \\
		id14llm \\
		id14llm@cs.umu.se \and \\
		Supervisor: Thomas Johansson\\
		Alexander Sutherland \\
		Thomas Hellström
}
\date{\today}

\renewcommand{\baselinestretch}{1.0}
\begin{document}
\maketitle

\newpage
\tableofcontents

\newpage
\section{Introduction}

This report contains the documentation for a perceptron based classification system
that guesses the emotional state of faces presented as input. The inputs are 20
x 20 pixel maps with 32 grey levels from white to black.

\subsection{Running the program}

The implementation is written in python 2.7 which means that a python 2.7 shell
is required to run the implementation. Required program arguments are a search
path to a \texttt{training.txt}, search path to an answer file
\texttt{training-facit.txt} and a test file \texttt{test-file.txt}. The format
of the training file should be the id of the image followed by the image on a
new line. The answer file should be formatted as the image id follow by the
correct answer (1,2,3 or 4).

Should the user fail to provide the required arguments, the program will
immediately terminate with a massage explaining that not enough input
arguments were provided.

Example usage of the program might look like:
\begin{verbatim}
python faces.py ../data/FaceTest/training-file.txt 
../data/FaceTest/training-facit.txt test-file.txt
\end{verbatim}


\bibliographystyle{abbrv}
\bibliography{rapport}

\section {Problem description}

The program works as such that it starts of by reading the \textit{images}. An
image consists of it's pixels (value from 0 - white to 31 - black) and it's ID
(sad, mischievous, happy, mad). The program then filters the noise and then
sends the images in the \textit{Tutor} class. The tutor uses Perceptrons to
learn what images look like (algorithm explained below). A perceptron is an
algorithm for learning a binary classifier. Afterwards the Examiner class makes
a guess and predicts which image belongs to which on a new set of image. The
program will then output the percentage of the guesses that were correct.


\subsection {Perceptron Learning Algorithm} 

The program implements an algorithm for supervised learning called
\textit{Perceptron Learning}. \textit{Supervised Learning} is a technuque for
machine learning where the Artificial neural network is exposed to input and
produces a guess based on the knowledge aquired from previous guesses. Since the
desired output from the ANN, in supervised learning, is known for every input,
the ANN is slightly modified with each iteration to eventually produce good
outputs for every set of inputs.

The ANN in the produced program is composed by four different Perceptrons
where each one represents one of the different facial expressions that are
to be identified in the assignment.

Each perceptron is implemented in \textit{Python} as a class composed by a
list of \textit{401} weights that mapps to the \textit{400} pixel values
that makes up an image. The additional weight is provided as a bias for
the perceptron. A bias is used in order to shift the result of the activation
function (more on that later) towards either the left or the right. I.e.
to make the output more extreme so that it will produce a more distinct
answer.

Training in the network occurs in the \textit{Tutor} class where the provided
set of training images is repetedly shown for each perceptron. The output
from a perceptron, also known as \textit{activation}, is computed using
the formula:

% Make pretty math notation here plz Marc
\begin{equation*}
  a_i = act(\sum_j x_j*w_{j,i})
\end{equation*}

where x is the input, w is the weight and act is the activation function,
which is the sigmoid function:

\begin{equation*}
	sigmoid(x) = \frac{1}{1+e^{-x}}
\end{equation*}


The difference between the expected output and the actual output is the
error $e$. The error e is then multiplied with the learning rate $\alpha$
and the pixel matrix to produce the gradient $\nabla w$. This
gradient is then added to every weight in the internal weight matrix for each
perceptron. This course of events can be described using the bellow formulas:

\begin{align*}
  &e_i = y_i - a_i\\
  &\nabla w_{j,i} = \alpha e_ix_j\\
  &w_{j,i} \leftarrow w_{j,i} + \nabla w_{j,i}
\end{align*}

This process is repeated for the entire set of training images, however,
only performing this process once would not produce very good results and
therefore, when the network has used all the training data, the images
are shuffled and the entire process described above is repeated.

The main problem encountered while training an ANN is finding the fine line
right before the network gets overfitted. When the ANN gets overfitted, it
becomes too specialized on the training data which will render it useless
when exposed to previously unseen data. It is, however, still crucial
to train the ANN as close to this limit as possible in order in order for
it to be able to make generalizations to the data and produce decent output.

One way of determine when the network has reached the desired level of training
is, for each iteration of the training data set, use a formula to compute the
sum of the output errors for the network and end training when the value from
this function gets below a predefined level. The function used in the current
implementation is:

\begin{equation*}
  E = \frac{1}{2}\sum^p_{i=1} ||y^{(i)} - d^{(i)}||^2
\end{equation*}

Where $p$ is the number of input/output vector pairs in the training set.

When the threshold for the error is reached, the perceptrons is deemed ready
to passed on to the \textit{Examiner} class where the network is exposed to
the set of test images. Each image is shown to every perceptron in the
network and the activation for each perceptron is recorded. The largest
activation is chosen as the answer for the network and is recorded into
our list of answers. When every image in the test set has been classified
the answers are printed to \textit{stdout}.

\section{Concluding discussion}

The biggest difficulty was choosing which programming language to use.
It was difficult choosing between Haskell and Python. Ultimately we chose
Python as our knowledge in Haskell was not sufficient to write this program.
It would be interesting to implement more AI and more specifically in machine
learning programs in a functional programming language as it's easier to deal with
concurrency. If you wanted to scale this program and have a larger neural
network it might be necessary to add concurrency.

\newpage
% This may not be the prettiest solution, but it's the one you deserve Linus
\lstinputlisting[language=Python]{../src/faces.py}
\lstinputlisting[language=Python]{../src/Enum.py}
\lstinputlisting[language=Python]{../src/Examiner.py}
\lstinputlisting[language=Python]{../src/Image.py}
\lstinputlisting[language=Python]{../src/Perceptron.py}
\lstinputlisting[language=Python]{../src/Tutor.py}
\lstinputlisting[language=Python]{../src/Utils.py}
\end{document}
